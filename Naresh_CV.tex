%%%%%%%%%%%%%%%%%%%%%%%%%%%%%%%%%%%%%%%%%
% Plasmati Graduate CV
% LaTeX Template
% Version 1.0 (24/3/13)
%
% This template has been downloaded from:
% http://www.LaTeXTemplates.com
%
% Original author:
% Alessandro Plasmati (alessandro.plasmati@gmail.com)
%
% License:
% CC BY-NC-SA 3.0 (http://creativecommons.org/licenses/by-nc-sa/3.0/)
%
% Important note:
% This template needs to be compiled with XeLaTeX.
% The main document font is called Fontin and can be downloaded for free
% from here: http://www.exljbris.com/fontin.html
%
%%%%%%%%%%%%%%%%%%%%%%%%%%%%%%%%%%%%%%%%%

%----------------------------------------------------------------------------------------
%	PACKAGES AND OTHER DOCUMENT CONFIGURATIONS
%----------------------------------------------------------------------------------------

\documentclass[a4paper,10pt]{article} % Default font size and paper size

\usepackage{fontspec} % For loading fonts
\defaultfontfeatures{Mapping=tex-text}
\setmainfont[SmallCapsFont = Fontin SmallCaps]{Fontin} % Main document font

\usepackage{xunicode,xltxtra,url,parskip} % Formatting packages

\usepackage[usenames,dvipsnames]{xcolor} % Required for specifying custom colors

\usepackage[big]{layaureo} % Margin formatting of the A4 page, an alternative to layaureo can be \usepackage{fullpage}
% To reduce the height of the top margin uncomment: \addtolength{\voffset}{-1.3cm}

\usepackage{hyperref} % Required for adding links	and customizing them
\definecolor{linkcolour}{rgb}{0,0.2,0.6} % Link color
\hypersetup{colorlinks,breaklinks,urlcolor=linkcolour,linkcolor=linkcolour} % Set link colors throughout the document

\usepackage{titlesec} % Used to customize the \section command
\titleformat{\section}{\Large\scshape\raggedright}{}{0em}{}[\titlerule] % Text formatting of sections
\titlespacing{\section}{0pt}{3pt}{3pt} % Spacing around sections

\begin{document}

\pagestyle{empty} % Removes page numbering

\font\fb=''[cmr10]'' % Change the font of the \LaTeX command under the skills section

%----------------------------------------------------------------------------------------
%	NAME AND CONTACT INFORMATION
%----------------------------------------------------------------------------------------

\par{\centering{\Huge R \textsc{Naresh}}\bigskip\par} % Your name
\par{\centering\large {\textsc{Undergraduate student at Indian Institute of Technology, Kharagpur}}\par}\large
%\par{{\begin{center}Dual Degree, \emph{Computer Science and Engineering}\end{center}}}
%\begin{flushleft}{\normalsize {\href{mailto:nareshmdu@gmail.com}{nareshmdu@gmail.com}}}\end{flushleft}
\hspace{2cm}\normalsize {\href{mailto:nareshmdu@gmail.com}{nareshmdu@gmail.com}}\hfill{+91-7872833729}\hspace{2cm}

%\section{Personal Data}
%
%\begin{tabular}{rl}
%\textsc{Place and Date of Birth:} & Canada  | 20 November 1987 \\
%\textsc{Address:} & 123 Broadway, City, State, Canada \\
%\textsc{Phone:} & +1 111 1112\\
%\textsc{email:} & \href{mailto:john@smith.com}{john@smith.com}
%\end{tabular}

%----------------------------------------------------------------------------------------
%	RESEARCH INTERESTS
%----------------------------------------------------------------------------------------

\section{Research Interests}

- Complex and Social Networks\hfill
- Autonomous Vehicles\hspace{3cm} \\
- Algorithms\hfill
- Machine Learning\hspace{3.6cm} \\
- Artificial Intelligence\hfill
- Cryptography\hspace{4.25cm} \\
- Software Design

%----------------------------------------------------------------------------------------
%	EDUCATION
%----------------------------------------------------------------------------------------

\section{Education}

\begin{tabular}{rl}	
2013-2018 & B.Tech and M.Tech (Dual Degree) in \textsc{Computer Science and Engineering}\\
\textsc{(Expected)}&\textbf{Indian Institute of Technology}, Kharagpur\\
%& 110/110 \small\emph{First Class Honours} | Major: Quantitative Finance\\
%& Thesis: ``Money is the Root of All Evil - Or is it?'' | \small Advisor: Prof. James \textsc{Smith}\\
&\normalsize \textsc{Cgpa}: 7.02/10.0\\
%\hyperlink{grds}{\hfill | \footnotesize Detailed List of Exams}\\
&\\

%------------------------------------------------

2013& Class XII, \textsc{}\textsc{Central Board of Secondary Education (CBSE)} \\
%&110/110 \small\emph{Commerce Specialization},
&\normalsize\textbf{Maharishi Vidya Mandir SSS}, Chennai\\
%& Heavily specialized in mundane paperwork | \small Advisor: Stefano \textsc{Bonini}\\
&\normalsize \textsc{Score}: 95.6\%\\
%\hyperlink{grds_usc}{\hfill| \footnotesize Detailed List of Exams}\\
&\\

%------------------------------------------------

2001 & Class X, \textsc{}\textsc{Central Board of Secondary Education (CBSE)} \\
&\normalsize\textbf{Kendriya Vidyalaya Picket}, Hyderabad\\
& \textsc{Gpa}: 10.0/10.0\\
%\hyperlink{grds_usc}{\hfill| \footnotesize Detailed List of Exams}\\
&\\

\end{tabular}


%----------------------------------------------------------------------------------------
%	Academic Projects
%----------------------------------------------------------------------------------------

\section{Academic Projects}

\begin{tabular}{r|p{11cm}}
%\emph{Current} & 1\textsuperscript{st} year Analyst at \textsc{Lehman Brothers}, London \\
\emph{Current} & \textbf{Kraken 2.0} \textsc{(Autonomous Mobile Robot)} \\
\textsc{Feb 2015} & \textbf{Group: }\textmd{\href{http://auv-iitkgp.in/}{Autonomous Ground Vehicle Research Group}}, IIT Kharagpur\\
& \textbf{Guide: }\textmd{\href{http://iitkgp.ac.in/fac-profiles/showprofile.php?empcode=aWmdU}{Professor C. S. Kumar}}\\
& \footnotesize{- Developing a robust autonomous mobile robot to participate in}\\
& \footnotesize{ the annual AUVSI ROBOSUB held in San Diego, California.}\\
& \footnotesize{- ROS (Robot operating system) has been adopted for software development and synchronization ROS works on a publisher-subscriber based architecture by means of passing messages. Built different packages on the ROS stack.}\\
\multicolumn{2}{c}{} \\

%------------------------------------------------

\emph{Current} & \textbf{Wikification via Link Co-occurrence} \\
\textsc{Feb 2015} & \textbf{Guide: }\textmd{\href{http://cse.iitkgp.ac.in/~pawang/}{Professor Pawan Goyal}}\\
& \footnotesize{- Wikification stands for the process of linking terms in a plain text document to Wikipedia articles which represent the correct meanings of the terms, can be thought of as a generalized Word Sense Disambiguation problem.}\\
\end{tabular}

%------------------------------------------------
%	Development Projects
%------------------------------------------------
\section{Development Projects}

\begin{tabular}{r|p{11cm}}

\textsc{Feb 2015} & \textbf{Advanced Graph Calculator} \\
& \footnotesize{- Contributed to the development of an advanced graph calculator that plots the graph of multi-variable systems on the screen.}\\
& \footnotesize{- The software was built on the PyQt framework using matplotlib and numpy. The PyQt framework was used for creating the GUI for the application. matplotlib was used for generating 2D/3D plots. numpy was used for explicitly generating domain for the functions to be plotted.}\\
\multicolumn{2}{c}{} \\

\textsc{Dec 2014} & \textbf{Lane Follower robot} \\
& \footnotesize{- Developed an autonomous lane following robot that uses advanced image processing techniques to detect the path in an IEEE certified workshop based on image processing organized by Technology Robotix Society, IIT Kharagpur}\\
\multicolumn{2}{c}{} \\

\textsc{Dec 2013} & \textbf{PIR Based Line Follower} \\
& \footnotesize{- Developed an autonomous line following robot using Atmel AVR microprocessor(Atmega16) in an IEEE certified workshop organized by Technology Robotix Scociety, IIT Kharagpur.}\\
\end{tabular}


%----------------------------------------------------------------------------------------
%	POSITIONS OF RESPONSIBILITY
%----------------------------------------------------------------------------------------

\section{Positions of Responsibility}

\begin{tabular}{rp{11cm}}
\emph{Current} & \textbf{Senior Editor}, Technology Literature Society, IIT Kharagpur \\
& \footnotesize{- Managing the content and design team of the society.}\\
& \footnotesize{- Writer in the English Team, and working as a senior editor for all English publications.}\\
\emph{Current} & \textbf{Secretary}, CodeClub, IIT Kharagpur \\
& \footnotesize{- Part of the managing team leading a group of 25 students.}\\
& \footnotesize{- Conducted several events, including Microsoft code.fun.do and BITWISE, the Annual Departmental Fest of the Department of Computer Science and Engineering, alongside several fortnightly competitive coding competitions within the campus.}\\
& \footnotesize{- Designed the software for updating the leaderboards in real-time during competitions as a part of BITWISE '15.}\\
\emph{Apr 2015} & \textbf{Team Member}, Google Students Club, IIT Kharagpur \\
& \footnotesize{- Organized multiple workshops and events, primarily focused on Android Development, in association with Google.}\\
& \footnotesize{- Conducted a workshop on the {\href{https://www.polymer-project.org/0.5/}{Polymer Project}}, which received high levels of}\\ & \footnotesize{participation.}\\
\emph{Apr 2014} & \textbf{Co-ordinator}, CodeClub, IIT Kharagpur \\
& \footnotesize{- Part of the organizing and sponsorship team for BITWISE '14.}\\
\end{tabular}

%----------------------------------------------------------------------------------------
%	SKILLS 
%----------------------------------------------------------------------------------------

\section{Computer Skills}

\begin{tabular}{r|p{11cm}}
\textsc{Proficient} & C, C++, Python, Robot Operating System (ROS), Java, mySQL\\
\multicolumn{2}{c}{} \\
\textsc{Dabbled} & HTML, CSS, Linux / Ubuntu, Git\\
\end{tabular}

%----------------------------------------------------------------------------------------
%	COURSEWORK
%----------------------------------------------------------------------------------------

\section{Coursework
\hfill\small\textsc{(T)heory and (L)aboratory}}

{\hspace{0.5cm}- Programming and Data Structures (T/L)} \hfill {- Algorithms-I (T/L)} \hspace{3.2cm} \\
{\hspace*{0.5cm}- Discrete Structures} \hfill {- Software Engineering (T/L)} \hspace{1.8cm} \\
{\hspace*{0.5cm}- Formal Languages and Automata Theory \hfill {- Switching Circuits (T/L)} \hspace{2.3cm} \\

%----------------------------------------------------------------------------------------

%\newpage
%----------------------------------------------------------------------------------------

\end{document}
